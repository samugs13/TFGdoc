% Resumen 
\newpage
\thispagestyle{plain}
\section*{Resumen}
\addcontentsline{toc}{chapter}{Resumen}

En este trabajo se lleva a cabo un análisis del estado del arte referente a herramientas para el despliegue de IaaS (Infrastructure as a Service) con el objetivo de realizar despliegues de de red heterogéneos virtualizados, aplicables a plataformas CyberRange para la formación y entrenamiento en el campo de la ciberseguridad. Centrándose este despliegue en la tecnología Terraform, con providers basados en Cloud y la tecnología de virtualización ligera Docker.

La idea principal es, a partir de la descripción de escenario a alto nivel (ficheros JSON, YAML, XML...), parametrizar y adaptar dichos ficheros a la tecnología de despliegue Terraform, evitando la dependencia asociada a los entornos físicos donde se desplegarán los escenarios de red virtualizados. El despliegue hace referencia tanto a la topología de red, interconexiones, como al aprovisionamiento de las máquinas finales. Una vez realizado el desarrollo, se manejará un catálogo de escenarios a virtualizar de forma automática, los cuales serán utilizados para la realización de ciberejercicios.

El código empleado en este trabajo está disponible en el siguiente repositorio de GitHub:

\vspace{0.2cm}

\begin{tcolorbox}[enhanced,attach boxed title to top center={yshift=-3mm,yshifttext=-1mm},
  colback=brown!20!white,colframe=orange!75!black,colbacktitle=orange!80!black,
  title=Dirección URL,fonttitle=\bfseries,
  boxed title style={size=small,colframe=orange!50!black} ]
  \centering
  \href{https://github.com/samugs13/DAERV}{\textbf{\color{blue}{https://github.com/samugs13/DAERV}}}
\end{tcolorbox}

\afterpage{\blankpage}
