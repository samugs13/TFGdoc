% Estado del arte
\chapter{Estado del arte}
\section{Tecnologías de virtualización}
Se podría decir que la virtualización es ya uno de los pilares fundamentales del mundo IT debido a las grandes ventajas que proporciona: escalabilidad, disponibilidad y una fácil gestión, lo que conlleva un ahorro importante en inversión en capital y gastos operativos (CAPEX y OPEX). \\

Previo al desarrollo de las tecnologías y tipos de virtualización disponibles, es conveniente explicar en qué consiste la virtualización, que no es más que una representación mediante software de un entorno físico o recurso tecnológico, como pueden ser aplicaciones, servidores o almacenamiento. \\ 

Una máquina virtual es un software que ejecuta programas \footnote{Un programa es un proceso en ejecución} como si fuera la máquina física, es decir, se abstrae el hardware y se representa con una capa de software. Por tanto, una VM \footnote{Siglas de "Virtual Machine"} proporciona una interfaz igual que el hardware, de forma que sobre ella podemos instalar uno o varios sistema operativos invitados o guests distintos. Los recursos del ordenador anfitrión se reparten entre las VM instaladas para abastecer sus necesidades. Cabe destacar que se produce un aislamiento de estos recursos que usan los invitados, lo que hace que el sistema anfitrión esté protegido si falla una VM, y que las VM estén protegidas entre ellas. 


\subsection{Virtualización con hipervisor}
\subsubsection{VirtualBox}
\subsubsection{VMWare}
\subsection{Virtualización en contenedores}
\subsubsection{LXC}
\subsubsection{Docker}
\section{Tecnologías de aprovisionamiento}
\section{Tecnologías de orquestación}
