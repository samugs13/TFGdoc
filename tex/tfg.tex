\documentclass[a4paper, 12pt]{report} % Formato de plantilla

\usepackage[utf8]{inputenc}
\usepackage[spanish]{babel}
\usepackage[margin=2cm, top=2cm, includefoot]{geometry}
\usepackage{graphicx} % Para la insercion de imagenes
\usepackage[table,xcdraw]{xcolor} % Para la deteccion de colores
\usepackage[most]{tcolorbox} % Para la insercion de cuadros
\usepackage{fancyhdr} % Para definir el estilo de la página
\usepackage[hidelinks]{hyperref} % Para gestionar hipervínculos
\usepackage{parskip} % Para arreglar la tabulación en el documento
\usepackage[backend=biber,style=numeric,sorting=none]{biblatex} % Para bibliografía 

% Declaracion de colores
\definecolor{color1}{HTML}{5E8BB4}

% Declaracion de variables
\newcommand{\logoportadaU}{../imgs/logos/ETSIT1.png}
\newcommand{\logoportadaD}{../imgs/logos/upm2.png}
\newcommand{\logopaginaD}{../imgs/logos/ETSIT2.png}
\newcommand{\grado}{Grado en Ingeniería de Tecnologías y Servicios de Telecomunicación}
\newcommand{\titulo}{Despliegue automatizado de escenarios de red virtualizados}
\newcommand{\fecha}{Junio 2022}
\newcommand{\autor}{Samuel García Sánchez}
\newcommand{\tutor}{Mario Sanz Rodrigo}

% Adicionales
\addto\captionsspanish{\renewcommand{\contentsname}{Índice}} % Cambio de formato del índice
\setlength{\headheight}{46.5423pt} % Definimos el margen de la barra horizontal de la cabecera
\pagestyle{fancy} % Definimos estilo fancy
\fancyhf{} % "Limpiamos" lo que trae el estilo fancy por defecto
\rhead{\includegraphics[width=1.5cm, keepaspectratio]{\logopaginaD}} % Foto de cabecera derecha
\lhead[\leftmark]{\rightmark} % Cabecera izquierda: capítulo en páginas pares y sección en impares
\cfoot{\thepage} % Número de página en el centro del pie de página
\renewcommand{\headrulewidth}{3pt} % Grosor de la barra de la cabecera
\renewcommand{\headrule}{\hbox to\headwidth{\color{color1}\leaders\hrule height \headrulewidth\hfill}} % Color de la barra de la cabecera
\bibliography{docs/refs} % Archivo .bib 

% Comienzo del documento
\begin{document}
	% Portada
\begin{titlepage}
\newpage
\changepage{2in}{}{}{-0.2in}{}{-0.6in}{}{}{}
\thispagestyle{empty}
\newpagecolor{orange}\afterpage{\restorepagecolor}
	
	\begin{center}

	\renewcommand{\baselinestretch}{2.0}
\textbf{\Large UNIVERSIDAD POLIT\'{E}CNICA DE MADRID}\\
\textbf{\large ESCUELA T\'{E}CNICA SUPERIOR \\ DE INGENIEROS DE TELECOMUNICACI\'{O}N}\\
	\vspace{1.0cm}
	\includegraphics[width=8cm]{\logoportadaU}
	\vspace{0.5cm} \\

\textbf{\Large \grado}\\
	\vspace{0.5cm}

\textbf{TRABAJO FIN DE GRADO}\\
	\vspace{3.5cm}
	\baselineskip=20pt

\textbf{\textbf{\Large \titulo}}\\
	\vspace{7.5cm}
	\baselineskip = 20pt

\textbf{\Large \autor} \\

\textbf{\Large \fecha} \\
	
	\end{center}

\newpage
\changepage{-0.5in}{}{}{0.2in}{}{0.5in}{}{}{}
\thispagestyle{empty}
\cleardoublepage
\end{titlepage}

	% ----------------------------------------------------------------------------------
        % TOC
        \tableofcontents
        %-----------------------------------------------------------------------------------
        % Sección Alumno
        \chapter{Resumen}
	\textbf{Nombre:} Tecnilógica Ecosistemas SAU \href{https://www.accenture.com/es-es/company-tecnilogica-accenture}{\textbf{\color{blue}Accenture}}
	\par a~\cite{buffett84} Hola ~\cite{watson53}
	\clearpage
	%----------------------------------------------------------------------------------
	% Bibliografía
	\printbibliography[heading=bibintoc]
\end{document}
