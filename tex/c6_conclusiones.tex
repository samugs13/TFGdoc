% Conclusiones y llíneas futuras
\chapter{Conclusiones y líneas futuras} \label{ch:con}
  Este capítulo expone las conclusiones extraidas a raíz de la realización del proyecto y las líneas futuras en las que se puede seguir trabajando.

\section{Conclusiones} \label{sec:concs}
  En el último año se ha experimentado un más que notable aumento del número de ciberataques y sofisticación de las amenazas conocidas. Es por esto que en la actualidad existe una gran demanda de perfiles de ciberseguridad. Este trabajo surge con la idea de hacer accesible y sencilla la práctica de ejercicios de tipo Red Team para la formación en ciberejercicios. 

  Con el desarrollo planteado en este trabajo de fin de grado, el cual ha alcanzado los objetivos \textit{O1}, \textit{O2}, \textit{O3} y \textit{O4} planteados en la tabla \ref{tab:objs} del capítulo \ref{ch:intro}, se ha demostrado que es posible el despliegue de escenarios de red virtualizados modelables mediante variables y configurables tanto a nivel de infraestructura como de software de forma sencilla y centralizada.

  El hecho de desplegar los escenarios en la nube es muy ventajoso respecto a los clásicos Cyber Range on-premises, pues los recursos se asignan bajo demanda, de forma que se paga sólo por lo que se usa y durante el tiempo que se usa, mientras que en un Cyber Range físico se ha realizado una inversión en equipos que hay que costear independientemente de si se usan o no. Además, la nube es fácilmente escalable, ya que se van asignando recursos según se van necesitando sin necesidad de interacción humana, mientras que en las instalaciones on-premises hay asociados unos costes de operación así como de tiempo en tener todo listo. El despliegue en cloud es también más seguro (es actualizado por el proveedor cloud con frecuencia, lo que conlleva también una reducción de costes en este aspecto) y configurable (permite una adaptación rápida y sencilla a cualquier topología o configuración software).

  Además, el uso de Terraform permite que este despliegue se realice de forma automatizada y en cuestión de segundos. Estos aspectos, junto con la fácil sintaxis, la portabilidad entre proveedores cloud, el uso de de ficheros de variables para parametrizar el despliegue y las funciones \texttt{file} y \texttt{templatefile} para ejecutar scripts de configuración personalizados en las instancias, hacen de Terraform una elección idónea para la orquestación de los escenarios.

  No obstante, el uso dichas tecnologías también derivó en algunos problemas, pero se encontraron soluciones alternativas, dando como resultado un sistema que cumple con los objetivos establecidos. Por ejemplo, en GCP, los scripts pasados a las instancias haciendo uso de la función \texttt{templatefile} se ejecutan como root antes de que haya ningún usuario creado ni logueado. Esto impide establecer variables de entorno en las instancias automáticamente en el arranque, de ahí la obligación de crear un segundo script para aprovisionar las instancias a través del proxy. Ambos scripts siguen la misma lógica, pero en este último, antes de cada orden fue necesario añadir el valor de la variable de entorno, que se pasaba como parámetro desde Terraform. Esto se puede apreciar con más detalle en el Anexo \ref{anx:aprov}. 

  A nivel personal, este trabajo me ha permitido adquirir conocimientos en tecnologías muy demandadas hoy en día y que son el futuro del sector por las grandes ventajas que proporcionan. 

\section{Líneas futuras} \label{sec:fut}
  Tras el desarrollo del proyecto, en esta sección se plantean posibles mejoras y líneas sobre las que trabajar en el futuro, con la intención de mejorar en la parametrización y configuración de los escenarios para hacerlo lo más intuitivo y fácil de usar posible.

  Este trabajo cubre los aspectos relacionados con el despliegue automatizado de los escenarios, así como el diseño y configuración de las conexiones entre los equipos. La principal línea a seguir sería la configuración software de dichos equipos, de forma que cumplan su papel en el escenario, obteniéndose así entornos funcionales en los que practicar tests de intrusión específicos. Dentro de esta configuración software distinguimos dos vías, una sería la configuración de servicios vulnerables mediante Docker, y la otra sería la creación de imágenes base personalizadas en GCP para los equipos de usuario.

  En cuanto a la configuración de los servicios, los ficheros que se proporcionan en el directorio \texttt{template-files} permiten aprovisionar las máquinas con contenedores Docker de forma sencilla tan sólo especificando algunos parámetros. De esta forma, es posible pasar imágenes vulnerables como las disponibles en el proyecto Vulhub~\cite{con1} como argumento a la instancia de Google Cloud a la hora de construirla, teniendo así un entorno funcional en el que practicar ejercicios de Red Team.

  Se plantea también como trabajo futuro la evolución de esta herramienta para que permita al usuario elegir en qué proveedor cloud desea desplegar los escenarios. Esto se conoce como capacidad multinube, y permitiría aumentar la disponibilidad y la flexibilidad, pudiendo elegir la nube que mejor se adapte a las necesidades concretas de cada escenario.
