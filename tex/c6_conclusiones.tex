% Conclusiones y llíneas futuras
\chapter{Conclusiones y líneas futuras} \label{ch:con}
  Este capítulo expone las conclusiones extraidas a raíz de la realización del proyecto y las líneas futuras en las que se puede seguir trabajando.

\section{Conclusiones} \label{sec:concs}
  En el último año se ha experimentado un más que notable aumento del número de ciberataques y sofisticación de las amenazas conocidas. Es por esto que en la actualidad existe una gran demanda de perfiles de ciberseguridad. Este trabajo surge con la idea de hacer accesible y sencilla la práctica de ejercicios de tipo Red Team para la formación en ciberejercicios. 

  Como se ha demostrado, es posible el despliegue de escenarios de red virtualizados modelables mediante variables y configurables tanto a nivel de infraestructura como de software de forma sencilla y centralizada.

  El hecho de desplegar los escenarios en la nube permite que cualquiera con permiso obtenga un escenario funcional donde practicar y formarse independientemente de dónde se encuentre y el sistema que emplee. Además, el hacerlo con Terraform permite que este despliegue se realice de forma automatizada y en cuestión de segundos. Estos aspectos, junto con la fácil sintaxis, la portabilidad entre proveedores cloud, el uso de de ficheros de variables para parametrizar el despliegue y la función \texttt{templatefile} para ejecutar scripts de configuración en las instancias hacen de Terraform una elección idónea para orquestar el despliegue.

  No obstante, el uso dichas tecnologías también derivó en algunos problemas, pero se encontraron soluciones alternativas, dando como resultado un sistema que cumple con los objetivos establecidos. Por ejemplo, los scripts pasados a las instancias haciendo uso de la función \texttt{templatefile} se ejecutan como root antes de que haya ningún usuario creado ni logueado. Esto impide establecer variables de entorno en las instancias automáticamente en el arranque, de ahí la obligación de crear un segundo script para aprovisionar las instancias a través del proxy. Ambos scripts siguen la misma lógica, pero en este último, antes de cada orden fue necesario añadir el valor de la variable de entorno, que se pasaba como parámetro desde Terraform. Esto se puede apreciar con más detalle en el Anexo \ref{anx:aprov}. 

  A nivel personal, este trabajo me ha permitido adquirir conocimientos en tecnologías muy demandadas hoy en día y que son el futuro del sector por las grandes ventajas que proporcionan. 

\section{Líneas futuras} \label{sec:fut}
  Tras el desarrollo del proyecto, en esta sección se plantean posibles mejoras y líneas sobre las que trabajar en el futuro, con la intención de mejorar en la parametrización y configuración de los escenarios para hacerlo lo más intuitivo y fácil de usar posible.

  La principal línea a seguir sería la configuración software de los equipos, de forma que se obtengan escenarios funcionales en los que practicar tests de intrusión específicos. Dentro de esta configuración software distinguimos dos vías, una sería la configuración de servicios vulnerables mediante Docker, y la otra sería la creación de imágenes base personalizadas en GCP para los equipos de usuario.

  En cuanto a la configuración de los servicios, como puede ser el servidor web en el escenario SCADA, los ficheros que se proporcionan en el directorio \texttt{template-files} permiten aprovisionar las máquinas con imágenes Docker vulnerables, como las disponibles en el proyecto Vulhub~\cite{con1}. De esta forma, es posible crear servicios con vulnerabilidades conocidas de forma personalizada mediante un Dockerfile, alojar las imágenes en DockerHub, y luego pasárselas como argumento a la instancia de Google Cloud a la hora de construirla, teniendo así un entorno funcional en el que practicar ejercicios de Red Team. 

  De igual forma, para los sistemas operativos base, como sería Kali Linux en el caso del atacante, se pueden crear imágenes en Google Cloud a partir de la ISO correspondiente y desplegar la instancia usando esa imagen pasándosela como parámetro de la misma forma que se hace con el contenedor.
